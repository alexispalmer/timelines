\section{Temporally-oriented possession} \label{sec:intro}

From a linguistic perspective, the term \textit{possession} refers to a particular set of semantic relations between two entities, the \textit{possessor} and the \textit{possessee} \cite{stassen2009predicative}.
The most typical notion of possession involves ownership or control of the possessee by the possessor, as in phrases like ``my piano,'' ``the lion's beautiful tail,'' or ``this friend of mine.''
A wide range of different asymmetric relationships fall under the heading of possession, including kinship, proximity, part-whole relations, experience of abstract concepts, and physical possession, both permanent and temporary. 
The linguistic literature makes a conceptual distinction between \textit{alienable possession}, in which possessees can be separated from their possessors, and \textit{inalienable possession}, in which such separation is not possible \cite[among others]{aikhenvald2012possession,heine2006possession}.

We are interested in alienable possession, and specifically in what we call \textbf{temporally-oriented possession} - changes in possession of an object over time.
% * <dhivyainfantchinnappa@my.unt.edu> 2018-01-11T02:23:45.056Z:
% 
% 'the phenomena'?
% 
% ^ <alexispalmer@gmail.com> 2018-01-11T02:46:35.612Z:
% 
% changed
%
% ^.
Temporality is an essential aspect of the linguistic discussion  of possession; in fact, the alienable/inalienable distinction can be viewed through the lens of temporality. 
Inalienable possessions are permanent possessions, while alienable possessions are temporary and, therefore, capable of changing hands.

\begin{figure}
\fbox{
\parbox[t]{3in}{\small Formerly a highlight of the \textbf{Ivan Morozov} collection in Moscow, the painting was nationalized and sold by the \textbf{Soviet authorities} in the 1930s. The painting was eventually acquired by \textbf{Stephen Carlton Clark}, who bequeathed it to the art gallery of \textbf{Yale University}.}
}
\caption{Text from Wikipedia article about Vincent Van Gogh's painting \textit{The Night Caf\'{e}}. Possessors marked in boldface.}\label{fig:nightCafe}
\end{figure}

The goal of this work is to facilitate a better understanding of temporally-oriented possession, in order to develop methods for automatic extraction of possession timelines from text. 
A possession timeline (see also Section~\ref{sec:timeline}) is an ordered presentation of the multiple possessors of some artifact.
Possession timelines could be used to enrich event timelines, potentially improving causal inference, as well as supporting text understanding. 
To this end, we present a new corpus of Wikipedia articles annotated with temporally-oriented possession information.

Previous work on possession in the NLP context has mostly focused on particular syntactic constructions.
\citet{conf/acl/TratzH13} investigate various semantic relations realized by English possessive constructions, and both \citet{Nakov:2013:SIN:2483969.2483975} and \citet{Tratz:2010:TDC:1858681.1858751} consider possession expressed in the form of noun compounds, such as ``family estate.''
We instead consider \textit{all} expressions of possession relations, whether phrasal, clausal, sentential, or even inter-sentential.
This non-restrictive approach is similar to that of \citet{BANEA16.1105}, who annotate possessions of particular bloggers at the time of utterance.
We instead focus on a particular set of artifacts, annotating their multiple possessors, as well as the order in which the artifacts pass from one party to another.
\citet{BANEA16.1105} include temporality in their annotations; each possession is typed as either temporary or permanent.

Our corpus is a set of Wikipedia articles about well-known artifacts which have been associated with different possessors over time; an excerpt of one article appears in Figure~\ref{fig:nightCafe}.
%: mostly paintings, but also some relics, diamonds, and archaeological discoveries. 
%All are artifacts which have changed hands many times, sometimes in notable ways.
In this data, both artifacts and their possessors are concrete entities, and possessors are limited to people, organizations, and locations.
We first extract all possession relations, in the form of artifact-possessor pairs, from each article, then annotate each relation with temporal features.
Finally, we arrange the possessors for a given artifact into a possession timeline.
 




