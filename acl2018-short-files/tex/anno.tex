\section{Annotating temporally-oriented possession}\label{sec:anno}

The annotation scheme was designed primarily to capture all temporal information relevant to changes of possession over time. 
In addition, we annotate whether particular features are certain or not, given the available textual evidence.


For consistency, annotators were provided with HTML pages of the 90 selected Wikipedia articles as downloaded on 12th June 2017. 
Annotation was done using the Wired-Marker\footnote{http://www.wired-marker.org/en} Firefox extension to annotate the HTML pages. 
First, all possessors of the target artifacts (Section~\ref{sec:possessors}) were highlighted, using different-colored markers (provided by Wired-Marker) for different named entity types (Section~\ref{sec:nes}). 
All other annotation features (Sections~\ref{sec:cert}, \ref{sec:temporal}, \ref{sec:timeline}) were added to the highlighted text using Wired-Marker's notes function.

The annotators were instructed to read the entire document to decide on the possessors and the order of possession. 
Unless a possessor possessed the artifact at different points in time, only one mention of each possessor is annotated. 



\subsection{Possessors and possessees/artifacts} \label{sec:possessors}
For each article, we focus on a single possessee/artifact, namely the topic of the article. 
For that artifact, we identify all possessors \textit{of that artifact} mentioned over the course of the article. 
The corpus consists of all artifact-possessor pairs identified from the selected Wikipedia articles. 
We extract 799 pairs in all, with 735 remaining after removing pairs which repeat within a single article. 
For example, the text snippet seen in Figure~\ref{fig:nightCafe} yields the following pairs:\footnote{The complete annotations for the article containing the text snippet are shown in Table~\ref{tab:timeline}.}

{\small
\begin{verbatim}
1. Night_Cafe-Ivan_Morozov
2. Night_Cafe-Soviet_authorities
3. Night_Cafe-Stephen_Carlton_Clark
4. Night_Cafe-Yale_University
\end{verbatim}
}


\subsection{The role of named entities} \label{sec:nes}
The possessors identified each fall into one of three named entity (NE) categories: Person, Organization (e.g. museums or universities), or Location (e.g. particular cities, states, or countries). 
The NE type of each possessor is labeled manually, with the distribution shown in Table~\ref{tab:anno-stats}. 
Organizations are the most frequent possessors, followed by People and then Locations. 
Although the possessors fall neatly into traditional NE categories, many of them are not in fact recognized by standard NE taggers. 
These include cases like (\ref{ex:storm}): the thieves who stole the painting in 1990, and presumably possessed it for at least some time thereafter, are unnamed.\footnote{In examples, possessors (and sometimes temporal anchors) in boldface; relevant Wikipedia article in brackets.} English NER systems may also struggle to recognize possessors such as ``artist's daughter'' or names in other languages.

\newcolumntype{x}[1]{>{\centering\arraybackslash}p{#1}}

\begin{table}[t]
\centering
%{\small
\begin{tabular}{|l|c|} \hline
\multicolumn{2}{|c|}{\textbf{All possessors}}\\ \hline \hline
Total for all articles & 799 \\ \hline
NE type: Pers / Org / Loc & 281 / 318 / 200 \\ \hline
Poss: Cert / Uncert & 774 / 25  \\ \hline
Time: Known / Unknown & \textbf{660} / 139  \\ \hline
Dur: Before / During / After & 7 / 647 / 6 \\ \hline
Dur: Cert / Uncert & 608 / 52   \\ \hline
\end{tabular}
%}
\caption{Statistics for all marked possessors} \label{tab:anno-stats}
\end{table}

% \begin{table*}[t]
% \centering
% \begin{tabular}{|l|x{5em}|x{5em}|x{5em}|} \hline
% & \multicolumn{3}{c|}{\textbf{Mentioned possessors}} \\ \hline \hline
% Total for all articles & \multicolumn{3}{c|}{799} \\ \hline
% NE type (Pers/Org/Loc) & 281 & 318 & 200 \\ \hline
% Certainty (Cert/Uncert) & 774 & 25 &  \\ \hline
% Time stamp (Known/Unknown) & \textbf{660} & 139 & \\ \hline
% Duration (Known/Unknown) & 608 & 52 &  \\ \hline
% Duration (Before/During/After) & 7 & 647 & 6 \\ \hline
% \end{tabular}
% \caption{Annotation statistics for all mentioned \& marked possessors} \label{tab:anno-stats}
% \end{table*}

\begin{table*}[t]
\centering
%{\small
\begin{tabular}{c|l||c|c|c|c|c} \hline
NE & Possessor & Poss.Cert & Order & Anchor & Duration & Dur.Cert \\ \hline \hline
PER & Vincent van Gogh & C & 1 & 1888 & During &  C \\ \hline
PER & Ivan Morozov & C & 2 & Unknown & - & - \\ \hline
PER & Stephen Carlton Clark & C & 4 & Unknown & - & - \\ \hline
ORG & Soviet authorities & C & 3 & 1930 & Before & C \\ \hline
ORG & Yale University & C & 5 & Unknown-Now & During & C \\ \hline
LOC & Moscow & C & 2 & Unknown & - & \\ \hline
LOC & New Haven, CT & C & 5 & Unknown-Now & During & C \\ \hline
\end{tabular}
%}
\caption{Complete annotation, including \textbf{possession timeline}, for Wikipedia article on Van Gogh's \textit{The Night Caf\'{e}}.} \label{tab:timeline}
\end{table*}

\begin{example}
On the morning of March 18, 1990, \textbf{thieves} disguised as police officers broke into the museum and stole \textit{The Storm on the Sea of Galilee} and 12 other works. {\small [WP: \textit{The Storm on the Sea of Galilee}]} \label{ex:storm}
\end{example}

\subsection{Certainty of possession} \label{sec:cert}
For each artifact-possessor pair, annotators are asked to assess the certainty of the possession relation. 
We are interested in the notion of certainty as it relates to textual evidence: if the text \textit{of the entire article} strongly supports the relation, the instance should be marked as Certain (C). 
If not, as Uncertain (UC). 
Nearly all relations are marked as certain (see Table~\ref{tab:anno-stats}).
Example (\ref{ex:genius}) illustrates a case of uncertainty - the phrase ``generally accepted'' indicates some degree of uncertainty on the part of the author.

\begin{example}
It was completed after Giorgione's death in 1510, with the landscape and sky generally accepted to have been completed by \textbf{Titian}. {\small[WP: \textit{The Sleeping Venus}]} \label{ex:genius}
\end{example}

\subsection{Annotating temporal location and duration} \label{sec:temporal}



The core of our approach is the annotation of temporal features of the extracted possession relations (i.e. the artifact-possessor pairs). 

\paragraph{Temporal anchor.}
The first time-related annotation decision is to determine whether, according to the text, there is a temporal anchor associated with the given possession relation. 
For cases when a possessor has held an artifact for more than one time period, different temporal anchors may be associated with the same artifact-possessor pair, as in \ref{ex:multiAnchor} below. 
This painting was in the custody of its owner prior to the 1873 Exhibition, and then again for a period between the end of that exhibition and the painting's 1878 journey out of Russia.

\begin{example}
Despite its progressive implications, Barge Haulers was bought by the \textbf{Tsar's second son}. It was lent for exhibition at the 1873 \textbf{International Exhibition} in \textbf{Vienna}, where it won a bronze medal. It was exhibited \textbf{outside Russia} again in 1878... {\small[WP: \textit{Barge Haulers on the Volga}]} \label{ex:multiAnchor}
\end{example}

% \begin{example}
% It was exhibited in the Uffizi Gallery for over two weeks and returned to the Louvre on 4 January 1914. {\small [WP: ???]} 
% \end{example}

% in case we need a shorter example for multiple anchors


\noindent
If a temporal anchor cannot be identified, the other temporal features are not relevant. 
Looking again at Figure~\ref{fig:nightCafe}, only one of the four possessors has an identifiable temporal anchor: \textbf{Soviet authorities} and \textbf{1930}.
In the corpus, 660 of 799 possessors are associated with a temporal anchor.



% \begin{example}
% The Birth of Venus (Italian: Nascita di Venere) is a painting by Sandro Botticelli probably made in the mid 1480s. {\small [WP: \textit{The Birth of Venus}]}
% \end{example}

\paragraph{Duration of possession.}
The temporal anchors identified in the previous annotation step are defined in a manner specific to this task. 
First, temporal anchors are marked at the level of the year; particular days or months are ignored. 

In the ideal case, the temporal anchor covers the entire period of possession (e.g. ``1983-1987''), but more often the text mentions a date or historical event (e.g. World War II) which may or may not lie within the duration of possession for a particular possessor.
In order to build possession timelines, we need to know how the temporal anchor relates to the period of possession. 
Thus we annotate three different categories of duration: Before (possession occurred prior to temporal anchor, (\ref{ex:before})), During (possession period includes temporal anchor, (\ref{ex:during})), and After (possession occurred after temporal anchor, (\ref{ex:after}).

\begin{example}
At some undetermined point before \textbf{1516} it came into the possession of \textbf{Don Diego de Guevara} ... {\small[WP: \textit{The Arnolfini Portrait}]} \label{ex:before}
\end{example}

\begin{example}
In \textbf{1599} a German visitor saw it in the \textbf{Alcazar Palace} in \textbf{Madrid}. {\small [WP: \textit{The Arnolfini Portrait}]} \label{ex:during}
\end{example}

\begin{example}
In \textbf{1530} the painting was inherited by Margaret's niece \textbf{Mary} of Hungary, who in 1556 went to live in Spain. {\small [WP: \textit{The Arnolfini Portrait}]} \label{ex:after}
\end{example}

\subsection{Possession ordering and timeline} \label{sec:timeline}

The final annotation task is to order the possessors according to when each had control of the artifact in question, building up a \textbf{possession timeline}. 
Each possessor is given a serial number, depending on the order in which the artifact was possessed. 
Usually the artist who created or found the artifact (if known) is assigned the serial number 1. 
The next possessor gets the serial number 2, and so forth.
An example timeline can be seen in Table~\ref{tab:timeline}.

Annotation relies on the complete textual context, often allowing the annotator to determine ordering of possession events even if explicit temporal anchors are not stated in the text. 
%This ordering annotation often provides possession timeline information even without an explicit statement of the year of a temporal anchor. 
For example, in Figure~\ref{fig:nightCafe}, we can easily infer that the possessor (\textbf{Yale University}) to whom the painting was bequeathed appears in the timeline later than the possessor who did the bequeathing.

When two possessors held the artifact simultaneously (e.g. \textbf{Yale University} and \textbf{New Haven, Connecticut}), both appear at the same position in the timeline. 
A possessor can receive multiple serial numbers when there are multiple relevant periods of possession (as in (\ref{ex:multiAnchor})), and repeated mentions of possessors tied to the same period of possession are marked as (e.g.) 1.1, 1.2, 1.3.

% \subsection{Dump of statistics}
% The numbers below indicate possessors that could be duplicates. That is, the same possessor can be found in more than one position in the timeline.
% Total number of possessors in all articles: 799
% Total number of possessors who are persons in all articles: 281
% Total number of possessors that are organizations in all articles: 318
% Total number of possessors that are locations in all articles: 200
% Total number of certain possessors in all articles: 774
% Total number of uncertain possessors in all articles: 25
% Total number of possessors with time unknown in all articles: 139
% Total number of possessors with time known in all articles: 660
% Total number of duration modifiers as 'Before' in all articles: 7
% Total number of duration modifiers as 'During' in all articles: 647
% Total number of duration modifiers as 'After' in all articles: 6
% Total number of possessors where the time duration is certain in all articles: 608
% Total number of possessors where the time duration is uncertain in all articles: 52
% Average number of possessors per article: 8.68
% Average number of possessors who are persons per article: 3.05
% Average number of possessors that are organizations per article: 2.17
% Average number of possessors that are locations per article: 3.45

% The following numbers correspond to unique possessors. That is, if the same possessor is found in more than one position in the timeline, only one of them is counted.

% Total number of possessors in all articles: 735
% Total number of possessors who are persons in all articles: 276
% Total number of possessors that are organizations in all articles: 168
% Total number of possessors that are locations in all articles: 291
% Average number of possessors per article: 7.989
% Average number of possessors who are persons per article: 3
% Average number of possessors that are organizations per article: 1.82
% Average number of possessors that are locations per article: 3.16


